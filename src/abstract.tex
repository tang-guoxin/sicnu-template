\begin{ChineseAbstract}
    自$2014$年起,CTex已不再维护和更新,然而现在流传下来的TeX模板仍然是基于CTex的模板。最新的一个版本是$2014$年的版本,最古老的可以追溯到$2004$年。TeXLive是未来的趋势。
    
    在过去的CTex模板中,仅仅是将模板元素的定义放到一个\verb|.tex|文件中。并且没有封装所有的定义和设置,在主程序中仍然有第三方包的导入和全局全局定义。这个模板将使用最简洁的代码来帮助作者完成你的毕业论文。只需要一行代码:\verb|\documentclass{sicnu}|。恭喜你,你已经完成了整个文档的所有设定,因为所有的元素都已被封装进了\verb|sicnu.cls|类中。

    这个模板封装了约$40$个包,新定义了$29$个可供直接调用的外部指令,新定义了$5$个的环境,重定义了$7$个环境。将来可能会对模板进行增删调补。并且分离了章节,而不是一股脑的放到一个文件中。这可以提高我们写作的专注力,同时便于检查错误和修改文章。或许听着有点多,但事实上这些命令和环境仅仅是帮助作者写入论文题目以及摘要关键词等信息,甚至你完全不需要阅读使用说明也能看懂这个指令做了什么。\textcolor{red}{你还可以通过预留的指令轻松将模板改成博士论文模板而不需要修改源代码。}

    最后,参考文献的引用和生成是区别于CTex模板的最大改进。现在文献将会根据引用的顺序在末尾自动排序(以前的模板需要手动排序)。并且使用了(GB/T 7714--2005)规范。

    \ChineseKeyword{关键词1;关键词2;关键词3;关键词4;关键词5}
\end{ChineseAbstract}

\begin{EnglishAbstract}
    Since $2014$, CTex is no longer maintained and updated, however, the TeX templates that have been handed down are still based on CTex. The latest version is the $2014$version, and the oldest version dates back to the $2004$year. TeXLive is the future.

    In the past CTex templates, merely defines the template elements into a \verb|.tex| file. And instead of encapsulating all the definitions and Settings, there are still third-party package imports and global global definitions in the main program. This template will use the most concise code to help the author complete your thesis. All it takes is one line of code: \verb|\documentclass{sicnu}|. Congratulations, you have completed the entire document set, because all of the elements have been encapsulated into the \verb|sicnu.cls| classes.

    This template encapsulates about $40$ packages, defines $29$ external instructions that can be called directly, defines $5$environments, and redefines $7$ environments. The template may be added or deleted in the future. And separate chapters instead of putting them all in one file. This improves our ability to focus on writing, and makes it easier to check for errors and revise. This may sound like a lot, but the fact is that these commands and environments just help the author write the title and abstract keywords, and you don't even need to read the instructions at all to understand what the command does. \textcolor{red}{You can also easily change the template into a doctoral thesis template with reserved instructions without changing the source code. }

    Finally, reference citations and generation are the biggest improvements that distinguish them from CTex templates. References will now be automatically sorted at the end by the order they are cited (previous templates required manual sorting). The specification (GB/T 7714--2005) is also used.

    \EnglishKeyword{Keyword1;Keyword2;Keyword3;Keyword4;Keyword5}
\end{EnglishAbstract}

