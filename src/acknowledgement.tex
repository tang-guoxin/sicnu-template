\begin{MyHeart}
    \par 古之学者必有师。师者,所以传道受业解惑也。人非生而知之者,孰能无惑?惑而不从师,其为惑也,终不解矣。生乎吾前,其闻道也固先乎吾,吾从而师之;生乎吾后,其闻道也亦先乎吾,吾从而师之。吾师道也,夫庸知其年之先后生于吾乎?是故无贵无贱,无长无少,道之所存,师之所存也。

    \par 嗟乎!师道之不传也久矣!欲人之无惑也难矣!古之圣人,其出人也远矣,犹且从师而问焉;今之众人,其下圣人也亦远矣,而耻学于师。是故圣益圣,愚益愚。圣人之所以为圣,愚人之所以为愚,其皆出于此乎?爱其子,择师而教之;于其身也,则耻师焉,惑矣。彼童子之师,授之书而习其句读者,非吾所谓传其道解其惑者也。句读之不知,惑之不解,或师焉,或不焉,小学而大遗,吾未见其明也。巫医乐师百工之人,不耻相师。士大夫之族,曰师曰弟子云者,则群聚而笑之。问之,则曰:“彼与彼年相若也,道相似也。位卑则足羞,官盛则近谀。”呜呼!师道之不复可知矣。巫医乐师百工之人,君子不齿,今其智乃反不能及,其可怪也欤!
    
    \par 圣人无常师。孔子师郯子、苌弘、师襄、老聃。郯子之徒,其贤不及孔子。孔子曰:三人行,则必有我师。是故弟子不必不如师,师不必贤于弟子,闻道有先后,术业有专攻,如是而已。
    
    李氏子蟠,年十七,好古文,六艺经传皆通习之,不拘于时,学于余。余嘉其能行古道,作《师说》以贻之。
\end{MyHeart}